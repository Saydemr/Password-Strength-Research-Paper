%%
%% This is file `sample-acmsmall-conf.tex',
%% generated with the docstrip utility.
%%
%% The original source files were:
%%
%% samples.dtx  (with options: `acmsmall-conf')
%% 
%% IMPORTANT NOTICE:
%% 
%% For the copyright see the source file.
%% 
%% Any modified versions of this file must be renamed
%% with new filenames distinct from sample-acmsmall-conf.tex.
%% 
%% For distribution of the original source see the terms
%% for copying and modification in the file samples.dtx.
%% 
%% This generated file may be distributed as long as the
%% original source files, as listed above, are part of the
%% same distribution. (The sources need not necessarily be
%% in the same archive or directory.)
%%
%% The first command in your LaTeX source must be the \documentclass command.
\documentclass[acmsmall,nonacm]{acmart}
\settopmatter{printacmref=false} % Removes citation information below abstract
\renewcommand\footnotetextcopyrightpermission[1]{} % removes footnote with conference information in first column
%\pagestyle{plain}

%%
%% \BibTeX command to typeset BibTeX logo in the docs
\AtBeginDocument{%
  \providecommand\BibTeX{{%
    \normalfont B\kern-0.5em{\scshape i\kern-0.25em b}\kern-0.8em\TeX}}}
    
\definecolor{beaver}{RGB}{215, 63, 9}
\definecolor{luminance}{RGB}{255, 181, 0}


\usepackage{amsmath}
\usepackage[mathletters]{ucs}
\usepackage[utf8x]{inputenc}
\usepackage{multicol}
\usepackage{comment}
\usepackage{graphicx}

%% Rights management information.  This information is sent to you
%% when you complete the rights form.  These commands have SAMPLE
%% values in them; it is your responsibility as an author to replace
%% the commands and values with those provided to you when you
%% complete the rights form.
%\setcopyright{none}
%\copyrightyear{}
%\acmYear{}
%\acmDOI{}

%% These commands are for a PROCEEDINGS abstract or paper.
%\acmConference[Woodstock '18]{Woodstock '18: ACM Symposium on Neural
%  Gaze Detection}{June 03--05, 2018}{Woodstock, NY}
%\acmBooktitle{Woodstock '18: ACM Symposium on Neural Gaze Detection,
%  June 03--05, 2018, Woodstock, NY}
%\acmPrice{15.00}
%\acmISBN{978-1-4503-XXXX-X/18/06}


%%
%% Submission ID.
%% Use this when submitting an article to a sponsored event. You'll
%% receive a unique submission ID from the organizers
%% of the event, and this ID should be used as the parameter to this command.
%%\acmSubmissionID{123-A56-BU3}

%%
%% The majority of ACM publications use numbered citations and
%% references.  The command \citestyle{authoryear} switches to the
%% "author year" style.
%%
%% If you are preparing content for an event
%% sponsored by ACM SIGGRAPH, you must use the "author year" style of
%% citations and references.
%% Uncommenting
%% the next command will enable that style.
%%\citestyle{acmauthoryear}

%%
%% end of the preamble, start of the body of the document source.
\begin{document}

%%
%% The "title" command has an optional parameter,
%% allowing the author to define a "short title" to be used in page headers.
\title{Password Strength Meters}

%%
%% The "author" command and its associated commands are used to define
%% the authors and their affiliations.
%% Of note is the shared affiliation of the first two authors, and the
%% "authornote" and "authornotemark" commands
%% used to denote shared contribution to the research.
\author{David Chen}
\email{chend2@oregonstate.edu}
\affiliation{%
  \institution{Oregon State University}
  \city{Corvallis}
  \state{Oregon}
  \country{USA}
  \postcode{97331}
}

\author{Abdullah Saydemir}
\email{saydemia@oregonstate.edu}
\affiliation{%
  \institution{Oregon State University}
  \city{Corvallis}
  \state{Oregon}
  \country{USA}
  \postcode{97331}
}

\author{Lindy Voss}
\email{vossli@oregonstate.edu}
\affiliation{%
  \institution{Oregon State University}
  \city{Corvallis}
  \state{Oregon}
  \country{USA}
  \postcode{97331}
}


%%
%% By default, the full list of authors will be used in the page
%% headers. Often, this list is too long, and will overlap
%% other information printed in the page headers. This command allows
%% the author to define a more concise list
%% of authors' names for this purpose.
\renewcommand{\shortauthors}{Chen, Voss \& Saydemir}

%%
%% The abstract is a short summary of the work to be presented in the
%% article.
\begin{abstract}
    \textcolor{beaver}{[ Abstract will be written here when the text is completed ]}
\end{abstract}

%%
%% The code below is generated by the tool at http://dl.acm.org/ccs.cfm.
%% Please copy and paste the code instead of the example below.
%%
\begin{CCSXML}
<ccs2012>
   <concept>
       <concept_id>10002978.10003029.10011703</concept_id>
       <concept_desc>Security and privacy~Usability in security and privacy</concept_desc>
       <concept_significance>500</concept_significance>
       </concept>
   <concept>
       <concept_id>10002978.10002991.10002992</concept_id>
       <concept_desc>Security and privacy~Authentication</concept_desc>
       <concept_significance>300</concept_significance>
       </concept>
   <concept>
       <concept_id>10002978.10003029</concept_id>
       <concept_desc>Security and privacy~Human and societal aspects of security and privacy</concept_desc>
       <concept_significance>500</concept_significance>
       </concept>
   <concept>
       <concept_id>10002978.10002979.10002983</concept_id>
       <concept_desc>Security and privacy~Cryptanalysis and other attacks</concept_desc>
       <concept_significance>100</concept_significance>
       </concept>
 </ccs2012>
\end{CCSXML}

\ccsdesc[500]{Security and privacy~Usability in security and privacy}
\ccsdesc[300]{Security and privacy~Authentication}
\ccsdesc[500]{Security and privacy~Human and societal aspects of security and privacy}
\ccsdesc[100]{Security and privacy~Cryptanalysis and other attacks}

%%
%% Keywords. The author(s) should pick words that accurately describe
%% the work being presented. Separate the keywords with commas.
\keywords{security, password, memory}

%% A "teaser" image appears between the author and affiliation
%% information and the body of the document, and typically spans the
%% page.

%%
%% This command processes the author and affiliation and title
%% information and builds the first part of the formatted document.
\maketitle

\section{Introduction}
Passwords are used nowadays as the first line of defense to protect information along with other options \cite{ma_2010}. Though passwords are meant to be solid and secure, they have flaws that can be easily exploited. In 2019, nearly 1 terabyte of user account data including passwords previously collected from Yahoo and some other website breaches were for sale \cite{hunt_2019}. Hashes of mostly used passwords, including those from Yahoo!, get uploaded to haveibeenpwned.com for the public to check whether their passwords were breached. Among 700 million unique passwords, two of the most commonly used were “123456” (more than 24 million times) and “password” (more than 3 million times) which can be broken in virtually zero time.


To prevent these from happening, the websites benefit from password composition rules (e.g. password must contain both lowercase and uppercase letters). Studies show that enforcing these rules helps users to choose stronger passwords \cite{shay_2010}. However, this also increases user fatigue since the average number of website accounts a single user has is around 25 \cite{lee_2014,waugh_2012}  and continues to increase. This exhaustion leads to people creating one (or a few) strong passwords, often reusing those passwords across many websites \cite{das_2014} and preserving them for long periods \cite{bhagavatula_2020,florencio_2007}. Moreover, only one third of people change their password after a breach and most of them change their passwords to weaker or equally strong ones \cite{bhagavatula_2020}. This leads to large collections of passwords (that are gathered on the internet) being used in brute force attacks over and over again, often with some success due to all of the unchanged passwords. There are also highly specific versions of dictionaries used in these attacks that are categorized according to attack type, platform, website, language and even to geological location.


The problem with strong and secure passwords is nobody can remember them. Especially with the amount of websites and accounts people hold nowadays. People are told to create strong passwords and in order to do that they have to avoid short passwords, birthdays, names, places, dictionary words, and the list goes on \cite{lee_2014}. Basically you are not supposed to use anything remotely memorable. Say you come up with a strong password like Jf5ruEf392HQlx and somehow memorize it. Well you can only use that for one account and the rest all have to be different but just as strong \cite{rubenking_2021}. This begs the big question of how do you create a strong password (for each account) that is realistic to memorize for practical use? There are tools and different tricks that have been found to work to help people balance these two very important factors when deciding on a password so they can not only remember their passwords, but also keep their information safe.

In this article, we propose the following contributions:

\begin{itemize}
\item We will have a look at password entropy calculation to understand what dictates the strength of passwords.
\item We introduce different password cracking techniques used to understand how passwords are being solved to better adapt new passwords to prevent cracking
\item We introduce password creation techniques and guidelines, and discuss their strengths and weaknesses (or flaws) in relation to password strength meters.
\item We introduce memorable passwords that are still strong to resist brute force / dictionary attacks.
\end{itemize}

In the next section, we will present an overview of concepts that are helpful to understand how a strong password is created.

\section{Concepts}

\subsection{Password} \label{password}

Password, common name for the term \emph{Memorized Secret Authenticator}, is a secret key that the user randomly selects and memorizes \cite{nist_2020}. Passwords must have enough complexity and secrecy that an adversary would be unable to guess or discover the secret value. If the password consists only of numbers it can be referred as PIN.

There are two other important definitions. \emph{Neutral Passwords} are those do not contain any information about the user whereas \emph{Biographic Passwords} are those that do contain information about the user \cite{Kavrestad_2019}.

\subsection{Passphrase} \label{passphrase}

Passphrase is a memorized secret that consists of sequences of words or phrases. It is usually longer than password to add more security \cite{passphrase_def}.

\subsection{Password Entropy / Strength} \label{entropy}
Password entropy is a measurement how strong a password is. It is used to estimate effectiveness of a password against guessing or brute-force attacks. Higher entropy means stronger passwords whereas lower entropy means weaker. It depends on the complexity and length \cite{hu_2018}.

To put it in mathematical way, let's say \emph{P} is a password of length \emph{l} and consists of characters in alphabet $\Sigma$ size of which is equal to \emph{N}. That is, \emph{N} = |$\Sigma$| . Then, the entropy \emph{E} of this password \emph{P} is calculated with the following equation :
\begin{equation} \label{eq:1}
E = \log_2 N^l = l \cdot \log_2 N
\end{equation}

We can directly deduce that increasing either of \emph{L} or \emph{N} creates stronger passwords. Therefore, to create a stronger password user should either choose a larger set of characters or make the password longer. It is obvious that for some specific numbers, entropy of longer password from a smaller set can be similar to entropy of the shorter password from a larger set. For example, let's consider two passwords. $\emph{P}_1$ consists of digits only and is of length 11. $\emph{P}_2$ consists of lowercase Latin letters and is of length 8. If we calculate the entropy of both using set sizes given in \autoref{tab:sets}:
\begin{align*}\label{eq:p1p2calc}
    \emph{E}_1 = \log_2 10^{11} &= 11 \cdot \log_2 10 = 11 \times 3.32 &\emph{E}_1 \approx 36.5 \\
    \emph{E}_2 = \log_2 26^8  &= 8  \cdot \log_2 26   = 8 \times 4.70 &\emph{E}_2 \approx 37.6
\end{align*}

\noindent we can see that there is not much difference between the entropy of the two passwords. However, it is important to note that entropy itself is \textbf{NOT} all it matters. Consider the same passwords $\emph{P}_1$ and $\emph{P}_2$ but this time let $\emph{P}_1$ be the string "46583159027" and $\emph{P}_2$ be the string "password". Though they have similar entropy, $\emph{P}_2$ is extremely weak and it can be broken in virtually zero time since it appears in leaked password lists.
\begin{center}
\begin{table}[ht!]

 \begin{tabular}[c]{| l  c  c |} 
\hline 
 Character Set & Elements & Set Size \\
 \hline\hline
 Digits & 0-9 & 10  \\ 
 \hline
 Latin letters (lowercase) & a-z & 26 \\
 \hline
 Latin letters (uppercase) & A-Z & 26  \\
 \hline
 Alphanumeric & a-z $\cup$  0-9 & 36  \\
 \hline
 Alphanumeric and Uppercase & a-z $\cup$ A-Z $\cup$ 0-9 & 62 \\  
 \hline
 Symbols (US keyboard) & $~!@\#\$\%\&*()-+=[]\{\}|;$ etc. & 32  \\
 \hline
 \end{tabular}
 \caption{Most Used Password Sets}
 \label{tab:sets}
\end{table}
\end{center}
Therefore, aside from the two parameters of \autoref{eq:1}, \emph{unpredictability}  is another factor that has an effect on password strength. We will talk about this concept in detail in \autoref{creation}.

Some adjustments can be made on password entropy calculation by making assumptions on the attackers strength. In this article, we will use the above method without any adjustments since assumptions may or may not hold true for each case. 

\subsection{Password Cracking-Techniques}

In order to know how to to best create a strong memorable password, it is important to know how many passwords are cracked. There are many different types of techniques that a hacker can use to figure out one's passwords. Each technique share the same end goal, but their execution differs.

\textcolor{beaver}{[ Generalized the methods into three categories: password-focused, human-focused, and computer-focused; this section will be focused on the password-focused, and human-focused ]}


\textcolor{beaver}{[ Go into how each technique might affect a certain password; how effective some are towards certain types of passwords v.s. other types; Lead into the next few examples of other techniques besides brute forcing ]}

\subsubsection{Brute Force}

\textcolor{beaver}{[ What is brute forcing? How it is related to entropy? ]}
\cite{pedamkar_2021}

\subsubsection{Dictionary Attack/Mask Attack}
\textcolor{beaver}{[ How this technique differs from brute forcing; What is the difference between Dictionary and Mask attacks? ]}
\cite{contributor_2005}
\cite{hashcat}

\subsubsection{Credential Stuffing}
\cite{mueller_wetter_manico_kingthorin}
\cite{imperva_2020}


\section{Password Creation Techniques} \label{creation}

We know what contributes to password strength  and how adversaries try to crack passwords. Hence, we can use these knowledge to create strong passwords. However, we need to know the essential requirements, dos and don'ts.

According to the latest NIST guidelines \cite{nist_2020}, a password must consist of at least 8 characters and users should be encouraged to use as lengthy password as possible, within reason. On the other side passwords should not include the following items:

\begin{itemize}
\item Passwords that are inside leaked password lists e.g. "password" 
\item Dictionary words e.g. "!-summer?5"
\item Repetitive or sequential characters e.g. "aaaaaaaa" or "abcd1234"
\item Context-specific words e.g. name of the service, username or derivatives
\end{itemize}

Having these in mind, let's see couple of methods to create strong passwords.

\subsection{Strong Passwords} \label{strong}

\subsubsection{Traditional Method} \label{creation-traditional}

There are several studies that show enforcing or suggesting password composition rules increase the password strength \cite{shay_2010, shay_2015}. Services usually provide compositions rules and often enforce the user to create a password that comply with these rules. The suggestions vary across services but most common ones are as follows:

\begin{itemize}
\item Password length should be at least 12
\item Password should include uppercase/lowercase letters, symbols and numbers
\item Password should not include obvious substitutions $S$ ${\displaystyle \rightarrow }$ $\$$
\end{itemize}

Prompted with these, what most users do is creating a sequence of string meaningful or meaningless and further changing (or adding) characters inside the string with symbols and numbers if necessary. Though this method, Traditional Method, is one of the most primitive methods used in password composition, it is still capable of offering strong passwords.


For example, let "\emph{latexcoroutine}" be the base string. A possible symbol and number substitution would be "$L/-\backslash t[-;CO2ine$". Therefore, the entropy of the password would be:
\begin{align*}
    \emph{E} = \log_2 94^{14} &= 14 \cdot \log_2 94 = 14 \times 6.55 &\emph{E} \approx 91.7 
\end{align*}



\begin{comment}
Traditional met
Traditional method refers to creating password based on the given requirements and/or suggestions on the screen during the password creation phase.
Though there are studies that shows enforcing or suggesting these rules increase the password
Since there is no consensus on suggestions, users create passwords that differ in length and complexity based on what the service requires/suggests. Variance in length and password directly affects the strength of password which means that created password are  
\end{comment}


\subsubsection{Passphrase Method} \label{creation-passphrase}

As we talked in \autoref{passphrase}, passphrases are lengthy passwords that consist of multiple words or phrases. \autoref{eq:1} specifies that increasing length of the password directly contributes to password entropy; therefore, increases the password strength. Furthermore, NIST suggests using passwords as lengthy as possible. Passphrase method tries to achieve highest password strength greedily increasing the length.

The words that form the passphrase must be random, varied,  and the passphrase must be neutral (i.e. non-biographic) \cite{nist_2020,Kavrestad_2019}. To accommodate the random word selection, Diceware \cite{reinhold_2021} offers a method and multiple wordlists consisting of more than 7000 words. Random passphrase creation method is as follows:

\renewcommand{\labelenumi}{\arabic{enumi}.}
\begin{enumerate}
    \item Roll a die five times (or five dices at once) and record the numbers 
    \begin{align*}
        \emph{4  2  4  6  4}
    \end{align*}
    \item Concatenate the result and find the corresponding word for your number in the wordlist 
    \begin{center}
        \includegraphics{samples/modish1}
    \end{center}
    
    \item Do this until you have sufficiently long passphrase (5-word is recommended)
    \begin{center}
        \emph{42464-63262-62165-22365-16432}\\
        \emph{modish-voss-turk-david-chen}
    \end{center}
\end{enumerate}


There are other wordlists proposed by Diceware or other platforms. By keeping the randomness of the procedure, this method can be applied to other wordlists as well. Furthermore, a secure random number generator could be used in order to create the random numbers faster.

One immediate drawback of the passphrase method is that NIST guidelines confront using dictionary words in a clear text format. Letter/number or letter/symbol substitutions should be made to avoid any possible password defects. However, know that common symbol substitutions such as $S$ ${\displaystyle \rightarrow }$ $\$$ does not increase the entropy \cite{ur_2015}.


\subsubsection{\textbf{Person-Action-Object (PAO) Method}} \label{creation-pao}

Person-Action-Object method is a password creation method first proposed and researched in detail by Carnegie Mellon University in 2013 \cite{blocki_2013}. Method was firstly used by mnemonists to recall certain objects

\subsection{Memorable Passwords}  \label{memorable}


The most secure passwords you could possible use would be a random string of letters, capitals, lowercase, numbers, and symbols that is as long as allowed \cite{lee_2014,nist_2020}. Those types of passwords, as secure as they may be, are nearly impossible to remember with the human memory, not to mention multiple of them. The human memory can only hold a sequence of about seven items and they cannot be arbitrary and should be redundant \cite{yan_2000}. That is the exact opposite of what a password is supposed to be. Creating passwords that are secure, yet memorable becomes very tricky. In order to do so, it is necessary to scale back the randomness, definitely the length, and therefore the secureness just enough, so the human memory has a chance of memorizing its passwords. You still want to preserve the security as much as possible though. It’s a balance that is hard to find, especially with each and every account password you have. Many are unable to find that balance and do the suggested security measures. For example, only 8 percent of users don’t reuse passwords \cite{lee_2014}.


The easiest way to create a memorable password is to use something important to you or something that you like. Just using the names, dates, and other words is not secure though \cite{rubenking_2021}. The best password you can create is the one you cannot remember which is why this is such a dilemma \cite{lee_2014}. Human memory, as it turns out, is the biggest flaw of passwords \cite{yan_2000}. So, we have to find a way to balance both of them. There are multiple techniques, tools, and rules that can be used to help like using something important as we mentioned or using outside management tools.


First, the rules to follow are the simplest way to increase the unbreakability without increasing the difficulty of memorability. Lengthening passwords is probably the easiest yet biggest thing that can increase security. Lengthening a password can be done by just adding some padding to the end. Make sure when you add padding though it is not just ‘!!!’ as that is common and easy to guess \cite{rubenking_2021}. Try adding 2 random alternating characters. According to a calculator for time to brute force Steve Gibson creating ‘helloworld!!!’ would take 33 years to crack but adding ‘qwqwqwqw’ puts it at over 49 trillion centuries \cite{rubenking_2021}. This way it has that element of random and the length, but it is not impossible to remember.


Another major rule to follow is avoiding names, places, and other common dictionary words \cite{lee_2014}. Even if there is not a lot of variety in type of characters in the password but is just not words and random characters instead, it will become more secure. This is because doing this takes dictionary attacks out of the equation forcing them to use a brute force. \cite{rubenking_2021}. Dictionary attacks crack about two-thirds of all passwords by using words and common patterns to guess like two numbers after a word or replacing letters for similar looking numbers \cite{lee_2014}. A password breach from Adobe has shown that some of the most common passwords (that you should avoid using) are these listed below \cite{lee_2014}:

\begin{center}
\begin{multicols}{2}
\begin{itemize}
    \item 123456
    \item 123456789
    \item password
    \item admin
    \item 12345678
    \item qwerty
    \item 1234567
    \item 111111
    \item photoshop
    \item 123123
    \item 1234567890
    \item 000000
    \item abc123
    \item 1234
    \item adobe1
    \item macromedia
    \item azerty
    \item iloveyou
    \item aaaaaa
    \item 654321
\end{itemize}
\end{multicols}
\end{center}


\noindent The dictionaries the attackers are using are almost guaranteed to contain these passwords, which is why forcing them to use a brute force attack is more ideal. Brute force attacks allow you to control your security much closer as they take more time to execute, and you are able to control the time they take to execute more directly with factors like character variation and length.
  

Adding in a variation of letters, capitalization, numbers, and symbols is also a good idea as well. This can exponentially increase the time it will take to crack as password just about as much as increasing length does even if the password is memorable and considered not strong according to typcial password creation rules\cite{lee_2014}. For example, according to Gibson’s calculator, ‘hellosworlds’ would take 16.54 minutes to crack as well as most combinations of 12 lowercase letters, where as ‘HeLlOwOrLd7?’ would take 1.74 centuries to crack \cite{rubenking_2021}. The simplest change can make a very big difference in your information's security even if you are sticking to the unsuggested memorable passwords.


So if we want to not use words and be random as suggested by these rules, how do we remember those passwords? Something completely random with no familiarization will be hard to remember. You can still create a password the appears completely random to an unsuspecting person, but has a pattern or meaning that the brain can attach to it so it can recall. As mentioned before use something meaningful. If you have a favorite movie, story, book, poem, song or even just a phrase, use that to your advantage \cite{lee_2014,rubenking_2021}. One of the many names for this method is the Bruce Schneier Method \cite{lee_2014}. This is when you take the first letter of each syllable (capitalized if it is the start of the word as well, otherwise lowercase) and the punctuation and abbreviate it into a random seeming password. For example, if someones favorite song is Carry on Wayward Son by Kansas they might use the chorus of that song. So "Carry on, my wayward son there'll be peace when you are done" would turn into "CrO,MWwST'lBPWYAD". This on its own appears completely random, but it is easier to remember because the brain can make a connection to something familiar. 


Even if you do not want to use this specific method to help remember passwords, any type of abbreviated password creation method will help the brain remember more easily. You can take lines from a TV show and add the season, episode, time or any other aspect to help randomize it as well \cite{rubenking_2021}. 


Some are unwilling to give up the security of a good password. For those people there are some ways those can still be memorable, although it is not as convenient as other listed methods. Using memorization techniques to remember passwords may be helpful. Things like mnemonic devices and phonetics in specific are proven methods \cite{rubenking_2021}. For example, say someone uses a random password generator to create 12 character passwords because they want their accounts to be as secure as possible. If it gives the password "Lf4Pwhw7haiR", they could use the mnemonic device "Love faded 4 Pheobe when he was 7 hours away in Riverdale". It will help the brain memorize the password while keeping the numbers, letters, and capitals straight. If they are trying to create one for each website, you could also try to incorporate that website or something similar into the device as well.


Another great memorization technique is phonetics. Trying to make words out of a password even if it is gibberish gives the brain a familiarity to hold on to and remember easier. For example if a generator outputs "drEnaba5Et" the phonetics might be pronounced as "Dr Enaba 5 E.T." \cite{rubenking_2021}. Another example is "orMSIZEfrH2O" as "or M size for H2O". This may prove to be difficult with completely random passwords so it may take some attempts to find one that works well enough to be realistic to use.


The last method for making passwords more realistic to remember is to actually not require memorable passwords at all. Instead use a password manager \cite{lee_2014}. This allows the person to not even have to bother with all the memorization and creation techniques and instead just focus on having completely secure passwords. There are many apps and websites that provide this service, or you can create your own in a way as well. For the absolute most secure passwords (including not reusing) this is the best option. All that is necessary is the memorization of one master password so there is continued access to rest of the passwords.


There is lots of other little tips and trick that can be done as well, but those were the best and most realistic ways to try to balance having the most secure password as possible and the limitations of the human memory.

\section{Results \& Analysis}
\textcolor{beaver}{[ In this section we will briefly talk about the previous sections and analyze the password creation technique we described, with examples ]} 


We analyzed several password creation methods. Each method offers different aspect to create passwords. Some helps us to create nearly unbreakable passwords while leaving memorability as a question, some takes different approach and puts memorability first. However, no method, including those that were not analyzed in this paper, provides both security and memorability  to get a user to remember 25 different passwords.Therefore, we suggest the following solution.

First, use an password manager to keep track of the passwords. There are robust applications on most of the platforms that provide this service besides some other useful features. Make sure that the application is secure and verified by other parties. Use one of the above methods to create single absolutely strong password and use it to secure the password manager itself. Enable two factor authentication if provided by the application.


\section{Contributions}
\begin{itemize}
\item{\verb|Lindy Voss       |} : Introduction, Memorable Passwords, References
\item{\verb|Abdullah Saydemir|} : Introduction, Concepts (\ref{passphrase} \ref{password} \ref{entropy}), Password Creation Techniques (\ref{strong}) References
\item{\verb|David Chen       |} : Introduction, Concepts
\end{itemize}


\section{Conclusion}



%%
%% The next two lines define the bibliography style to be used, and
%% the bibliography file.
\bibliographystyle{ACM-Reference-Format}
\bibliography{psm_bibliography}

%%
%% If your work has an appendix, this is the place to put it.
\appendix


\end{document}
\endinput
%%
%% End of file `password_strength.tex'.
